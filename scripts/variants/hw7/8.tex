%UNIQ:MAP%

<<Отображение>>~-- Map. Класс должен хранить набор пар (ключ, значение), причём ключи
уникальны. Тип ключа и тип значения должны быть параметрами шаблона. Написать
несколько конструкторов, в том числе конструктор копирования.

В классе реализовать
методы для добавления нового элемента в отображение, удаления элемента, поиска по
ключу и вывода на экран. Перегрузить операции для добавления нового элемента в
отображение (сложение), удаления элемента по ключу (вычитание), объединения
(сложение), пересечения (умножения), индексирования, присваивания и вывода на экран.

Создать массив объектов и передавать пары объектов в функцию, которая строит
отображение, состоящее из элементов, входящих только в одно из заданных отображений,
т.е. $(A\bigcup B) \setminus (A\bigcap B)$, и возвращает его в основную программу.

Реализовать предложенный шаблонный класс (тип элементов, которые
хранит класс~-- параметр шаблона; каждый объект класса может хранить
элементы только одного типа). Продемонстрировать реализованную
функциональность класса для работы с различными типами данных.

Исключительные ситуации необходимо обрабатывать через механизм исключений.
