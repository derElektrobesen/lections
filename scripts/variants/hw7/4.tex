
<<Вектор>>~-- Vector размерности n. Размерность вектора~-- также параметр шаблона. Класс
должен содержать несколько конструкторов, в том числе конструктор копирования. В
классе реализовать методы для вычисления модуля вектора, скалярного произведения,
сложения, вычитания, умножения на константу.

Перегрузить операции сложения,
вычитания, умножения векторов, умножения на константу, инкремента, декремента,
индексирования, присваивания и вывода на экран.

Создать массив объектов. Написать
функцию, которая для заданной пары векторов будет определять, являются ли они
коллинеарными или ортогональными.

Реализовать предложенный шаблонный класс (тип элементов, которые
хранит класс~-- параметр шаблона; каждый объект класса может хранить
элементы только одного типа). Продемонстрировать реализованную
функциональность класса для работы с различными типами данных.

