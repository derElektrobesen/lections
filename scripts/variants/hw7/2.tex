%UNIQ:BTREE%

<<Упорядоченное бинарное дерево>>~-- Tree. Тип хранимых в дереве ключей~-- параметр
шаблона. Написать несколько конструкторов, в том числе конструктор копирования.

В
классе реализовать методы для вставки элемента в дерево, удаления элемента по ключу,
поиска элемента по ключу, вывода на экран. Перегрузить операции сложения, вычитания,
присваивания и вывода на экран.

Cформировать дерево, вывести содержимое его узлов в
порядке возрастания, найти узел по ключу, вывести содержимое листьев дерева (вершин,
не имеющих потомков).

Реализовать предложенный шаблонный класс (тип элементов, которые
хранит класс~-- параметр шаблона; каждый объект класса может хранить
элементы только одного типа). Продемонстрировать реализованную
функциональность класса для работы с различными типами данных.


Исключительные ситуации необходимо обрабатывать через механизм исключений.
