%UNIQ:COMPLEX%

<<Комплексное число>>~-- Complex. Типы действительной и мнимой части должны быть
параметрами шаблона. Класс должен содержать несколько конструкторов.

В классе
реализовать методы для сложения, вычитания, умножения и вывода на экран. Перегрузить
операции для сложения, вычитания, умножения, присваивания и вывода на экран.

Создать
два вектора размерности n из комплексных координат. Передать их в функцию, которая
выполняет сложение комплексных векторов.

Реализовать предложенный шаблонный класс (тип элементов, которые
хранит класс~-- параметр шаблона; каждый объект класса может хранить
элементы только одного типа). Продемонстрировать реализованную
функциональность класса для работы с различными типами данных.

