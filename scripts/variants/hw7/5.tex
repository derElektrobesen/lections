
<<Стек>>~-- Stack. Элементы стека хранятся в массиве. Если массив имеет фиксированную
размерность, то предусмотреть контроль выхода за пределы массива. Если память
выделяется динамически и ее не хватает, то увеличить размер выделенной памяти.

В
классе реализовать методы для включения элементов в стек и их извлечения из стека.

Перегрузить операции сложения, вычитания, присваивания и вывода на экран. Создать
массив объектов. Передавать объекты в функцию, которая удаляет из стека первый
(сверху), третий, пятый и т. д. (нечётные) элементы.

Реализовать предложенный шаблонный класс (тип элементов, которые
хранит класс~-- параметр шаблона; каждый объект класса может хранить
элементы только одного типа). Продемонстрировать реализованную
функциональность класса для работы с различными типами данных.

