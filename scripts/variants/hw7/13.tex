
<<Квадратная матрица>>~-- Matrix. Размерности матрицы~-- также параметры шаблона.
Класс должен содержать несколько конструкторов, в том числе конструктор копирования.
В классе реализовать методы для сложения, вычитания, умножения матриц; вычисления
нормы матрицы.

Перегрузить операции сложения, вычитания, умножения, присваивания и
вывода на экран.

Создать массив объектов класса Matrix и передать его в функцию,
которая изменяет i-ю матрицу путем возведения ее в квадрат. В основной программе
вывести результат.

Реализовать предложенный шаблонный класс (тип элементов, которые
хранит класс~-- параметр шаблона; каждый объект класса может хранить
элементы только одного типа). Продемонстрировать реализованную
функциональность класса для работы с различными типами данных.
