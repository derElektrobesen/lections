%UNIQ:POLYNOM%

<<Многочлен>>~-- Polynom степени n. Степень полинома также параметр шаблона.
Написать несколько конструкторов, в том числе конструктор копирования.

В классе
реализовать методы для вычисления значения полинома; сложения, вычитания и
умножения полиномов. Перегрузить операции сложения, вычитания, индексирования,
присваивания, вывода на экран.

Создать массив объектов класса. Передать его в функцию,
вычисляющую сумму полиномов массива и возвращающую полином-результат, который
выводится на экран в основной программе.

Реализовать предложенный шаблонный класс (тип элементов, которые
хранит класс~-- параметр шаблона; каждый объект класса может хранить
элементы только одного типа). Продемонстрировать реализованную
функциональность класса для работы с различными типами данных.
