
<<Множество>>~-- Set (элементы уникальны). Написать несколько конструкторов, в том
числе конструктор копирования. В классе реализовать методы для определения
принадлежности заданного элемента множеству, пересечения, объединения, разности двух
множеств.

Перегрузить операции сложения, вычитания, умножения (пересечения),
индексирования, присваивания и вывода на экран. Создать массив объектов и передавать
пары объектов в функцию, которая строит множество, состоящее из элементов, входящих
только в одно из заданных множеств, т.е. $(A\bigcup B) \ (A\bigcap B)$, и возвращает его в основную
программу.

Реализовать предложенный шаблонный класс (тип элементов, которые
хранит класс~-- параметр шаблона; каждый объект класса может хранить
элементы только одного типа). Продемонстрировать реализованную
функциональность класса для работы с различными типами данных.

