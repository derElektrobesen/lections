Реализовать базовый класс Item (единица хранения в
		библиотеке), содержащий данные-члены:
invNumber~-- инвентарный номер и taken~--
взято на руки или имеется в наличии, а также методы:
\begin{description}
	\item{\Code{virtual void Show();}} //показать информацию о единице
		хранения
	\item{\Code{bool isAvailable();}} // есть ли единица хранения в наличии ?
\item{\Code{int GetinvNumber();}} //возвращает инвентарный номер
\item{\Code{void Take();}} // операция <<взять>>
\item{\Code{void Return();}}// операция <<вернуть>>
\end{description}

Построить производные классы Book и Magazin.
Класс Book содержит данные-члены: \Code{author}, \Code{title}, \Code{publisher},
\Code{year}
и методы: \Code{Author();} \Code{Title();} \Code{Publisher();} \Code{YearOfPublishing();} \Code{Show()}.

Класс Magazin включает данные-члены: \Code{volume;} \Code{number;} \Code{year;} \Code{title}
и методы: \Code{Volume();} \Code{Title();} \Code{Number();} \Code{Year();} \Code{Show()}.

Создать массив указателей на объекты базового класса и
заполнить этот массив объектами производных классов.
Вызвать метод \Code{Show()} базового класса и просмотреть массив
объектов.
