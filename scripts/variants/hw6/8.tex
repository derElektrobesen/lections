Реализовать базовый класс <<Вычисление определённого
интеграла>>, содержащий чисто виртуальный метод для
вычисления значения интеграла (которому в качестве
		параметра должен передаваться указатель на целевую
		подынтегральную функцию), поля-данные задающие начало
и конец отрезка интегрирования, а также число отрезков
разбиения.

Создать производные классы <<Метод трапеций>> и
<<Метод парабол>>, реализующие соответствующие численные
методы.

Метод для вычисления значения интеграла в
базовом классе должен осуществлять разбиение фигуры под
графиком подынтегральной функции на соответствующие
части, а потом собирать результаты вычисления площадей
этих частей на основе результатов функций, реализованных
в производных классах.

Продемонстрировать работу
реализованных методов.
