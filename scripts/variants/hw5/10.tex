%UNIQ:POLYNOM%
Реализовать класс \textbf{<<Многочлен>>~-- Polynom}

В классе степени $n$ перегрузить операции сложения, вычитания,
умножения, инкремента, декремента, индексирования, присваивания, вывода на экран.

Создать массив объектов класса. Передать его в функцию, вычисляющую сумму
полиномов массива и возвращающую полином-результат, который выводится на экран в
основной программе (использовать при этом реализованные перегруженные операции).
