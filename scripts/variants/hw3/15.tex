\section*{Домашнее задание №3}

%UNIQ:VECTOR%

Написать программу, которая в качестве аргумента командной строки
принимает имя текстового файла, содержащего элементы трёх видов:
\begin{description}
\item{$+$} <слово>
\item{$-$} <слово>
\item{$?$} <слово>
\end{description}

Элементы отделяются друг от друга одним или несколькими
разделителями~-- пробелами, табуляциями, символами новой строки.

Слово
с предшествующим плюсом добавляется в упорядоченный динамический
массив, если его там ещё нет (в качестве функции сравнения слов
использовать лексикографическое сравнение). Если числу предшествует
минус, то это слово удаляется из списка (если оно было в нём). Если перед
словом стоит вопрос, то оно печатается в выходной поток в отдельной
строке вместе со словом \Code{Yes} или \Code{No} в зависимости от того, присутствует ли
это слово в построенном на тот момент списке.
