\section*{Домашнее задание №3}

%UNIQ:B_TREE%

Написать программу, которая в качестве аргументов командной строки
принимает имена двух текстовых файлов. В первом файле содержится
последовательность целых чисел.

Необходимо записать во второй файл
числа, содержащиеся в первом файле, упорядоченные в порядке
возрастания.

При реализации алгоритма необходимо использовать
двоичное дерево поиска (ключами при построении дерева должны
являться сами числа). Числа в первом файле разделяются символами, для
которых библиотечная функция \Code{isspace()} возвращает ненулевое значение.
