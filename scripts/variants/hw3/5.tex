\section*{Домашнее задание №3}

%UNIQ:STACK%

Написать программу, которая в качестве аргумента командной строки
принимает имя текстового файла, содержащего строку символов, в
которой записано выражение со скобками трёх типов: $[$ $]$, $\{$ $\}$, $($ $)$.

Необходимо определить, правильно ли расставлены скобки (не учитывая
остальные символы) и напечатать на стандартный выходной поток
результат.

При реализации алгоритма необходимо использовать стек.
Например, в выражении $[()]\{\}$ скобки расставлены правильно, а в
выражениях $][$ и $[(\{)]\}$ неправильно.
