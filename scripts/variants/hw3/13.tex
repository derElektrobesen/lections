\section*{Домашнее задание №3}

%UNIQ:B_TREE%

Написать программу, которая в качестве аргумента командной строки
принимает имя текстового файла, содержащего элементы трёх видов:

\begin{description}
\item{$+$} <неотрицательное целое число>
\item{$-$} <неотрицательное целое число>
\item{$?$} <неотрицательное целое число>
\end{description}

Элементы отделяются друг от друга одним или несколькими
разделителями~-- пробелами, табуляциями, символами новой строки.

Число
с предшествующим плюсом добавляется в двоичное дерево поиска, если
его там ещё нет (ключом при построении дерева должно являться само
число). Если числу предшествует минус, то это число удаляется из дерева
(если оно было в нём). Если перед числом стоит вопрос, то оно печатается
в выходной поток в отдельной строке вместе со словом \Code{Yes} или \Code{No} в
зависимости от того, присутствует ли это число в построенном на тот
момент дереве поиска.
