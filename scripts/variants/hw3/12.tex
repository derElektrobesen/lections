\section*{Домашнее задание №3}

%UNIQ:B_TREE%

Написать программу, которая в качестве аргументов командной строки
принимает имена двух текстовых файлов. В первом файле содержится
последовательность слов.

Необходимо записать во второй файл слова,
содержащиеся в первом файле, упорядоченные в порядке убывания.

При
реализации алгоритма необходимо использовать двоичное дерево поиска
(ключами при построении дерева должны являться сами слова, в качестве
функции сравнения слов использовать лексикографическое сравнение).

Слова в первом файле разделяются символами, для которых библиотечная
функция \Code{isspace()} возвращает ненулевое значение.
