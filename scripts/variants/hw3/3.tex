
%UNIQ:DL_LIST%

Написать программу, которая в качестве аргумента командной строки
принимает имя текстового файла, содержащего слова.

Необходимо
вывести на стандартный выходной поток список всех различных слов в
файле в порядке убывания количества вхождений слов (с указанием
количества вхождений).

При реализации алгоритма необходимо
использовать двусвязный список, каждый узел которого должен
содержать указатель на слово и количество вхождений этого слова.

Слова
во входном файле разделяются символами, для которых библиотечные
функции \Code{isspace()} или \Code{ispunct()} возвращают ненулевое значение.
