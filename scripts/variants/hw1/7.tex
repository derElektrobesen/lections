\section*{Домашнее задание №1}

Написать программу, которая в качестве аргументов командной строки
принимает имена текстовых файлов (документов), а на выходной поток
выдаёт различные слова в этих документах с указанием частоты
встречаемости слов в документах.

Слова должны выводиться в порядке
убывания их частоты встречаемости. При одинаковой частоте
встречаемости выдавать слова в лексикографическом порядке. Частоту
встречаемости считать как отношение количества документов, в которых
встречается данное слово, к общему количеству документов.

Слова во
входных файлах разделяются символами, для которых библиотечные
функции \Code{isspace()} или \Code{ispunct()} возвращают ненулевое значение. Считать,
что словарь различных слов можно хранить в оперативной памяти.
