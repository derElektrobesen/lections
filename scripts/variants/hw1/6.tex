
Написать программу, которая в качестве аргумента командной строки
принимает имя текстового файла (документа), а на выходной поток выдаёт
различные слова в этом файле с указанием их частоты встречаемости в
тексте документа.

Слова должны выводиться в порядке убывания их
частоты встречаемости. При одинаковой частоте встречаемости выдавать
слова в лексикографическом порядке. Частоту встречаемости считать как
отношение количества вхождений слова к общему количеству слов в
тексте документа.

Слова во входном файле разделяются символами, для
которых библиотечные функции \Code{isspace()} или \Code{ispunct()} возвращают
ненулевое значение. Считать, что словарь различных слов можно хранить в
оперативной памяти.
