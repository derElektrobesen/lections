\section*{Домашнее задание №1}

Написать программу, которая в качестве аргументов командной строки
принимает заданное слово (первый аргумент) и имена текстовых файлов
(документов). Необходимо для этого слова посчитать дисперсию его
вхождений в документы:
$$
	D=\sqrt{ \frac{ \sum^N_{i = 1}{ (a_i-\acute{a})^2 } }{ N - 1 } }
$$

Где $D$~-- дисперсия, $a_i$~-- количество вхождений слова в $i$-ый документ,
$N$~-- общее количество документов,
$\acute{a}$~-- среднее арифметическое вхождений слова по всем документам:
$$
	\acute{a} = \frac{ \sum^N_{i=1}{a_i} }{N}
$$

Слова во входных файлах разделяются символами, для которых
библиотечные функции \Code{isspace()} или \Code{ispunct()} возвращают ненулевое
значение.
