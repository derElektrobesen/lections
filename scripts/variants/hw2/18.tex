
В аргументе командной строки передаётся имена текстовых файлов
в которых записаны двумерные матрицы
вещественных чисел.

Необходимо напечатать на стандартный выходной
поток результат следующего алгоритма:

Евклидова норму матрицы, находящейся во входном файле

$$
\sqrt{ \sum_i\sum_j{|a_{ij}|^2} }
$$

\textbf{Формат хранения матриц в файлах}~--
разреженный~-- в первой строке файла записано количество строк
матрицы; во второй~-- количество столбцов матрицы; в третьей~--
количество ненулевых элементов матрицы; в последующих строках
файла записаны индексы и значения этих ненулевых элементов
матрицы (информация о каждом элементе матрицы записана на
отдельной строке) в следующей последовательности: \Code{номер\_строки}
\Code{номер\_столбца} \Code{значение\_элемента}. Считать, что пары индексов
элементов матрицы в файле упорядочены по возрастанию.

В качестве внутреннего представления матрицы в памяти программы
использовать двумерный массив вещественных чисел.

Целевой алгоритм не
должен зависеть от представления матрицы в файле и в памяти, т.е. для
работы с матрицей должны быть реализованы и использованы в алгоритме
функции: \Code{double get\_elem(void *matr, int row, int col)}, которая возвращает
значение элемента матрицы по его индексам, и \Code{void set\_elem(void *matr, int
row, int col, double elem)}, которая устанавливает значение элемента
матрицы по его индексам
